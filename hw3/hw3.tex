\documentclass[draft]{article}
\usepackage{amsmath}
\usepackage{mathbbol}
\usepackage{mathtools}
\usepackage[letterpaper,top=1in,bottom=1in,left=1in,right=1in]{geometry}
\usepackage{chngcntr}
\usepackage{amssymb}
\counterwithin*{equation}{section}
\counterwithin*{equation}{subsection}

\title{CS 135 Written Assignment 3}
\author{Daniel Detore}
\date{February 17, 2024}

\begin{document}

\maketitle

\section{}
\begin{align}
     & \neg r                    & $Hypothesis$                  & \\
     & q \implies r              & $Hypothesis$                  & \\
     & \neg q                    & $Modus tollens, 1,2$          & \\
     & \neg q \implies u \land s & $Hypothesis$                  & \\
     & u \land s                 & $Modus ponens, 3,4$           & \\
     & s                         & $Simplification, 5$           & \\
     & p \lor q                  & $Hypothesis$                  & \\
     & p                         & $Disjunctive syllogism, 3, 7$ & \\
     & p \land s                 & $Conjunction, 6, 8$           & \\
     & p \land s \implies t      & $Hypothesis$                  & \\
     & t                         & $Modus ponens, 9,10$          &
\end{align}

\section{}

For simplicity's sake, I will use this key to represent the given propositions:
\begin{itemize}
    \item $p$: The dorm is locked.
    \item $q$: The phone is on top of the tall bookshelf.
    \item $r$: The dorm room is locked.
    \item $s$: The phone is under the pillow.
    \item $t$: The dorm has more than 10 floors.
    \item $u$: The phone is in the bottom drawer of the desk.
\end{itemize}

This gives us the following list of statements:

\begin{itemize}
    \item[] $p \implies \neg q$
    \item[] $r \implies q$
    \item[] $p$
    \item[] $r \lor s$
    \item[] $t \implies u$
\end{itemize}
Using this list, we can deduce the following:
\begin{align}
     & p                 & $Hypothesis$                 & \\
     & p \implies \neg q & $Hypothesis$                 & \\
     & \neg q            & $Modus ponens 1, 2$          & \\
     & r \implies q      & $Hypothesis$                 & \\
     & \neg r            & $Modus tollens 3, 4$         & \\
     & r \lor s          & $Hypothesis$                 & \\
     & s                 & $Disjunctive syllogism 5, 6$ &
\end{align}
We have now confirmed that $s$ must be true. If we check the key, this means that 
the phone is under the pillow.

\section{}
Let us start at the contradiction form $x = \neg x$ and derive the given formula 
from it. I will use $E$ to replace $x$, and introduce new variables in reverse
alphabetical order.
\begin{align*}
    & E \lor \neg E & $Original proposition$& \\
    & \equiv (D \lor E) \land (\neg D \lor E) \land \neg E& $Resolution$ & \\
    & \equiv (B \lor E) \land (\neg B \lor D) \land (\neg D \lor E) \land \neg E& $Resolution$ & \\
    & \equiv (C \lor B) \land (\neg C \lor E) \land (\neg B \lor D) \land (\neg D \lor E) \land \neg E& $Resolution$ & \\
    & \equiv (A \land B) \land (\neg A \lor C) \land (\neg C \lor E) \land (\neg B \lor D) \land (\neg D \lor E) \land \neg E& $Resolution$ & \\
    & \equiv (A \land B) \land (\neg A \lor C) \land (\neg B \lor D) \land (\neg C \lor E) \land (\neg D \lor E) \land \neg E& $Commutative law$ &
\end{align*}

\section{}
To make this argument form invalid, its premises need to be true and its
conclusion needs to be false. For simplicity's sake I will make $P(x)$ and
$Q(x)$'s domains $\{T,F\}$ and they will return the value I give them.
\begin{align*}
     & \forall x( P(x) \implies Q(x)) \\
     & \neg P(a)                       \\
     & \therefore \neg Q(a)
\end{align*}
Because $\forall x( P(x) \implies Q(x))$ places the same $x$ value into both $P(x)$ and $Q(x)$, we can simplify it to $x \implies x$. This statement is a tautology. Using the same logic, $P(a) \implies Q(a)$ (the statement implied by argument form) will be equivalent to $a \implies a$ which is, again, a tautology. Since I have proven all of the statements to be tautologies, this argument form is always true.

\section{}

\section{}
This argument is invalid. If we look past the logic at what argument is being made,
the conclusion is always false.



\end{document}
% latexmk -pvc -pvctimeoutmins=5 -pdf -pdflatex="pdflatex -interaction nonstopmode" hw3.tex

Let us define some placeholder predicate names.
\begin{align*}
    T(x) &$: $ x $ can run 10 km in $<30$ minutes.$\\
    S(x) &$: $ x $ is a smoker.$\\
    H(x) &$: $ x $ can run 100 m in $<11$ seconds.$
\end{align*}
The domain of all three predicates consists of all athletes. Here is the argument
rewritten with these placeholders:
\begin{align*}
    & \neg \exists x(T(x) \land S(x)) \\
    & \neg \exists x(S(x) \land H(x)) \\
    & \therefore \forall x( T(x) \implies \neg H(x))
\end{align*}