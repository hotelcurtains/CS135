\documentclass{article}
\usepackage{amsmath}
\usepackage{mathbbol}
\usepackage{mathtools}
\usepackage[letterpaper,top=1in,bottom=1in,left=1in,right=1in]{geometry}
\usepackage{chngcntr}
\usepackage{amssymb}
\usepackage[verbose]{placeins}
\counterwithin*{equation}{section}
\counterwithin*{equation}{subsection}
\renewcommand{\thesubsection}{\thesection.\alph{subsection}}

\DeclarePairedDelimiter{\floor}{\lfloor}{\rfloor}
\DeclarePairedDelimiter{\ceil}{\lceil}{\rceil}

\title{Written Assignment 5}
\author{Daniel Detore\\CS135-B/LF}
\date{March 5, 2024}

\begin{document}
\maketitle

\section{}
\subsection{}
\begin{gather*}
    f = \{ (a, 1), (b, 1), (c, 2) \} \\
    f = \{ (a, 1), (b, 2), (c, 1) \} \\
    f = \{ (a, 1), (b, 2), (c, 2) \} \\
    f = \{ (a, 2), (b, 1), (c, 1) \} \\
    f = \{ (a, 2), (b, 1), (c, 2) \} \\
    f = \{ (a, 2), (b, 2), (c, 1) \}
\end{gather*}
All of these sets of pairs are functions and their domains contain all elements 
of $B$, therefore they are all onto functions $f: A \rightarrow B$.

\subsection{}
\begin{gather*}
    f = \{ (a, 1), (b, 1), (c, 1) \} \\
    f = \{ (a, 2), (b, 2), (c, 2) \}
\end{gather*}
All of these sets of pairs are functions but their domains do not contain all 
elements of $B$, therefore none of them are onto functions $f: A \rightarrow B$.

\subsection{}
There are no one-to-one functions $f: A \rightarrow B$ because $|A| > |B|$. As such
it is impossible to map each element of $A$ uniquely onto an element of $B$.

\section{}
\subsection{}
We can prove that $P$ is one-to-one by proving that, if two inputs provide the same output, than the inputs must be equal.\\
Let us define two strings $a, b \in X$. To prove that $P$ is one-to-one, we can find that if $P(a) = P(b)$ then $a = b$. 
If $P(a) = P(b)$, then $1a = 1b$.
Since these strings are equal, removing both of their first character will still result in equal strings. This leaves us with $a = b$. 
This proves that in all cases where $P(a) = P(b)$ then it must be true that $a = b$. As such, $P$ is a one-to-one function from $X$ to $X$.

\subsection{}
A simple counterexample is 0. Any bit string that starts with 0 is also a counterexample.
\begin{itemize}
    \item $P$'s domain and target are both $X$, the set of all bit strings, as per its definition. 
    \item $X$ contains strings that begin with both 1 and 0. 
    \item $P$ guarantees that the first element of any string in its range will be 1. 
    \item Any element in $X$ that starts with 0 is outside of the range of $P$.
    \item $P$ cannot be onto because there are elements in its target that are not in its domain.
\end{itemize}

\section{}
We need to prove:
\begin{equation} 
    \floor*{ \frac{\floor*{\dfrac{x}{3}}}{2} } = \floor*{ \frac{x}{6} }
\end{equation}
Let us start by making a placeholder for the left side of the equation. Let
\begin{equation} 
    \floor*{ \frac{\floor*{\dfrac{x}{3}}}{2} } = \alpha.
\end{equation}
Table 1, on page 159 of our textbook \emph{Discrete Mathematics and Its 
Applications}, offers the following biconditional:
\begin{equation} \tag{Table 1.(1a)}
    \floor{x} = n \text{ if and only if } n \leq x < n+1
\end{equation}
We can apply (Table 1.(1a)) to (2) to acquire the following:
\begin{equation*}
    \alpha \leq \frac{\floor*{\dfrac{x}{3}}}{2} < \alpha + 1
\end{equation*}
We can use algebra and multiply each part of this compound inequality by 2 to
remove that denominator:
\begin{equation*}
    2\alpha \leq \floor*{\dfrac{x}{3}} < 2\alpha + 2
\end{equation*}
Table 1 also offers the following compound inequality:
\begin{equation*} \tag{Table 1.(2)}
    x-1 < \floor{x} \leq x \leq \ceil{x} < x+1
\end{equation*}
We can shave this down to the relevant inequality:
\begin{equation}
    \floor{x} \leq x
\end{equation}
Using (3), we can reasonably assume the following:
\begin{gather}
    2 \alpha \leq \floor{\dfrac{x}{3}} \implies 2 \alpha \leq \dfrac{x}{3} \\
    \floor{\dfrac{x}{3}} < 2\alpha +2 \implies \dfrac{x}{3} < 2\alpha + 2
\end{gather}
Using the right sides of (4) and (5), we get the following compound inequality:
\begin{equation*}
    2 \alpha \leq \dfrac{x}{3} < 2\alpha + 2
\end{equation*}
Dividing all parts by 2 gives us the following:
\begin{equation}
    \alpha \leq \dfrac{x}{6} < \alpha + 1
\end{equation}
We can now apply (Table 1.(1a)) to (6):
\begin{equation*}
    \floor*{\dfrac{x}{6}} = \alpha
\end{equation*}
And we can fill in the value of $\alpha$:
\begin{equation*}
    \floor*{\dfrac{x}{6}} = \floor*{ \frac{\floor*{\dfrac{x}{3}}}{2} }
\end{equation*}
If we switch the sides, we reach (1), which was to be proven:
\begin{equation} \tag{1}
    \floor*{ \frac{\floor*{\dfrac{x}{3}}}{2} } = \floor*{ \frac{x}{6} }
\end{equation}



\end{document}






onto functions missing some element of A:
\{ (a, 1), (b, 2) \} \\
\{ (a, 2), (b, 1) \} \\
\{ (a, 1), (c, 2) \} \\
\{ (a, 2), (c, 1) \} \\
\{ (b, 1), (c, 2) \} \\
\{ (b, 2), (c, 1) \} \\

NOT onto functions missing some e of X:
\{ (a, 1) \} \\
\{ (a, 2) \} \\
\{ (b, 1) \} \\ 
\{ (b, 2) \} \\
\{ (c, 1) \} \\
\{ (c, 2) \} \\
\{ (a, 1), (b, 1) \} \\
\{ (a, 1), (c, 1) \} \\
\{ (b, 1), (c, 1) \} \\