\documentclass{article}
\usepackage{amsmath}
\usepackage{mathbbol}
\usepackage{mathtools}
\usepackage[letterpaper,top=1in,bottom=1in,left=1in,right=1in]{geometry}
\usepackage{chngcntr}
\usepackage{amssymb}
\usepackage{graphicx}
\usepackage[verbose]{placeins}
\counterwithin*{equation}{section}
\counterwithin*{equation}{subsection}
\renewcommand{\thesubsection}{\thesection.\alph{subsection}}

\title{Written Assignment 4}
\author{Daniel Detore\\CS135-B/LF}
\date{February 25, 2024}

\begin{document}

\maketitle

\section{}
\subsection{}
32 is not a square square-free number. It is divisible by 4, which is $ 2^2 $. 

\subsection{}
This is an invalid way to use the superset operation. John is trying to compare an integer with a set, which is not possible. He should have written:
\begin{equation*}
P (\{17, 19, 22, 26, 30\}, \{30\})
\end{equation*}


\section{}
The given set $U$ is the set of all perfect square numbers.

\subsection{}
Yes. One such set is $B = \{0, 1, 4, 9, 16, 25, 36, 49\}$. $|B| = 8$, so $|B| > 7$.

\subsection{}
Yes. One such set is $B = \{64\}$. In this case, $B \subseteq U$ and $10 \leq 64 \leq 100$.

\end{document}